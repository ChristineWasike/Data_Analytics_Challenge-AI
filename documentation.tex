\documentclass[12pt,a4paper]{article}
\usepackage{graphicx}
\graphicspath{ {./img/} }
\usepackage[paper=a4paper]{geometry}
\usepackage[utf8]{inputenc}
\usepackage{pdfpages}

\usepackage{todonotes}

\usepackage{ragged2e,array,url,graphicx,csquotes,hyperref,adjustbox,longtable}
\usepackage[english]{babel}
\usepackage[
backend=biber,
style=ieee,
citestyle=ieee
]{biblatex}

\usepackage{setspace}

\renewcommand{\topfraction}{.85}
\renewcommand{\bottomfraction}{.7}
\renewcommand{\textfraction}{.15}
\renewcommand{\floatpagefraction}{.66}
\renewcommand{\dbltopfraction}{.66}
\renewcommand{\dblfloatpagefraction}{.66}
\setcounter{topnumber}{9}
\setcounter{bottomnumber}{9}
\setcounter{totalnumber}{20}
\setcounter{dbltopnumber}{9}

\addbibresource{references.bib} 
\defbibheading{bibliography}[\refname]{}


\begin{titlepage}
    \begin{center}
        \vspace*{1cm}
            
        \Huge
        \textbf{Assignment 2}
            
        \vspace{0.5cm}
        \LARGE
        Data Analytics Challenge
            
        \vspace{1.5cm}
            
        \textbf{Cohort 2 Group 1}
            
        \vspace{1.2cm}
        
        \Large 
        Joyce Njeri, Christine Wasike, Alice Fatmata\\
        Ayo Oluwapamilerin, Gilbert Sibomana
        
        \vspace{8.0cm}
            
        \Large
        Department: Computer Science\\
        University: African Leadership University\\
        Country: Rwanda\\
        Date: \today
            
    \end{center}
    
\end{titlepage}

\begin{document}


\doublespacing
\tableofcontents
\singleplacing

\vspace{15cm}


\section{Introduction}

\subsection{Purpose}

This document is Group 1 Cohort 2's bid to Rwanda Revenue Authority (RRA), which explains the implementation of our prototype, the environment we need to scale the project to production, our team’s budget, timelines, risk, and recommendations for the successful implementation of the project.

\subsection{Document Convention}

This document follows the Latex format. 

\section{Stakeholders}

\subsection{Client}

\begin{itemize}
\item Name: 
Rwanda Revenue Authority (RRA)
\item Address: 
Kimihurura, Kigali, Rwanda
\item Phone Number: 
(250) 788 185 500
\item Email Address: 
info@rra.gov.rw
\end{itemize}

\subsection{Contractor}

\begin{itemize}
\item Name: 
1C2(Cohort 2 Group 1)
\item Address: 
KG 126 St, Kigali, Rwanda
\item Phone Number: 
(250) 784 650 219
\item Email Address: 
info@alueducation.com
\end{itemize}

\section{Prototype}

\subsection{Background Information}

The aviation industry is substantial for every country’s revenue system as it is the gateway to getting people in and out of the country. One of the biggest challenges in the aviation industry is that it is costly to maintain a good standard due to financial hurdles and legal complications [2]. 

In 2015 Virgin America, a US-based airline, experienced customer loss and profit loss when the government issued a license tax on special fuel(airplane fuel). This made special fuel expensive for Virgin America, who decided not to increase ticket fares for customers but to cut costs. Unknowing to them, their cost-cutting affected their services, and they ended up losing customers [5]. 

Not only did they lose customers, but they also had their image tarnished as customers took it to Twitter to air their displeasure [2].

\subsection{Implementation}

\subsubsection{Data Collection}

We researched using the prompt to know the kind of data that is needed to solve the problem. After our research, we downloaded data that we found relevant to solving the problem. We used Pandas’ to import the data set to Jupyter Notebook, our Artificial Intelligence environment. After importing it to Jupyter Notebook we used exploratory data analysis (EDA) to analyze, investigate and summarize the main characteristics of the data set [1].

\subsubsection{Data Preparation}

There are different ways to handle missing values: from dropping them, to filling them with zero, the mean, mode, or the median [2]. Slicing is the process of selecting specific rows and columns of data based on some criteria. We used it in data preparation to highlight the unique rows and columns we wanted to work with [2].

\subsubsection{Data Transformation}

Due to long column names, we used the Pandas’ rename method to get short and simplified titles.


\subsubsection{Data Integration}

We integrated the sentiment data set with the financial data set in order to draw correlations and inferences.

\subsubsection{Data Visualization}

We have used Matplotlib [4] and Seaborn libraries in Python to visualize the data and to find a trend in the tax revenues collected by the United States government [1].

\section{Environment}

\subsection{Hardware}

\begin{itemize}
\item Computer or a Laptop 
\end{itemize}

\subsection{Software}

\begin{itemize}
\item Jupyter Notebook
\item Python libraries 
\item Operating System of your choice
\end{itemize}

\section{Administration}

\subsection{Budget}

Below is the estimated manpower budget, assuming that software is open-sourced and computer hardware is available to the client.

\begin{table}[h!]
\begin{tabular}{ | p{3.5cm} | p{2cm} | p{3.5cm} | p{6cm} |  }
 \hline
 Expenses & Cost(Per month / USD) & Role Breakdown & Description \\
 \hline
 Data Analyst' Quotation & 5000  & Head Analyst Data Analyst &
Money to pay the Data Analysts working on the product prototype back-end.\\
 
 Software Development Quotation & 9000 & Head Developer Other Developers & Money allocated towards developing the product, i.e., Mobile Application, and the back-end system\\
 \hline
 \end{tabular}
 \caption{Budget Breakdown}
\label{table:1}
\end{table}

\subsection{Timelines}

Below is the estimated project timeline.

\begin{table}[h!]
\begin{tabular}{ | p{4cm} | p{8cm} | p{4cm} |  }
 \hline
 Months & Tasks	& Status \\
 \hline
 February & Data Preparation & Completed \\
 March & Prototype Creation	& Completed \\
 April	& Implementing And Testing Prototype & Not Started \\
 May	& Building Final Product & Not Started \\
 June & Product Release & Not Started \\
 \hline
 \end{tabular}
 \caption{Project Timelines}
\label{table:2}
\end{table}

\subsection{Risk and Mitigation}

Lack of adequate data may result in inaccurate findings. Having more data provides a more substantial basis for making fact-based decisions that can better inform company strategies.

To mitigate the risk of inadequate data, the institution must maintain a steady stream of data supply through customer collection.

\section{Recommendations}

We recommend Virgin America to increase their ticket prices and also restructure their target market to a customer demographic that can afford it.

As an alternative to increasing prices, we recommend that they should ensure proper allocation of funds to departments that interact directly with the customer base. 

\section{Conclusion}

With the data set of our choice, we have been able to demonstrate how Exploratory Data Analysis (EDA) is the best way to understand and summarise the characteristics and behaviors of the data. 

In summary:

\begin{itemize}
\item Plotting techniques help validate the hypothesis which is made about data.
\item EDA helps us to understand which model will fit best for predictions about the data set.
\item EDA also reduces our efforts at the time of machine learning model building
\end{itemize}
 
\textbf{References}

\begin{enumerate}
    \item [1] What is Exploratory Data Analysis?, Ibm.com, 2020. [Online]. Available: https://www.ibm.com/cloud/learn/exploratory-data-analysis. [Accessed: 04- Mar- 2021].
    
    \item [2] Part 4: Data Management and Analysis, Reporting and Disseminating Results. WHO STEPS Surveillance, 2017, pp. 4-1-1 to 4-4-1.

    \item [3] M. Wood, Python and Matplotlib Essentials for Scientists and Engineers, 3rd ed. Quebec: Morgan & Claypool, 2015, p. 150pp.

    \item [4] "What’s new in each version — seaborn 0.11.1 documentation", Seaborn.pydata.org, 2021. [Online]. Available: https://seaborn.pydata.org/whatsnew.html. [Accessed: 05- Mar- 2021].
    
    \item [5] Z. Luvsandorj, "Simple word cloud in Python", Towards Data Science, 2020. [Online]. Available: https://towardsdatascience.com/simple-wordcloud-in-python-2ae54a9f58e5. [Accessed: 05- Mar -2021].

\end{enumerate}
\printbibliography

\end{document}